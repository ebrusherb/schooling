\documentclass{article}
%\documentclass[aps, twocolumn, tightenlines,amscd,amsmath,amssymb,verbatim]{revtex4}
\usepackage{latexsym}
\usepackage{amssymb,amsmath}
\usepackage{custom2}
\usepackage{graphicx} % for figures
\usepackage{epstopdf} % so can use EPS or PDF figures
%\usepackage{subfig}
\usepackage{caption}
\usepackage{subcaption}
\usepackage{url}
\usepackage{amssymb,amsfonts}
\usepackage[all,arc]{xy}
\usepackage{enumerate}
\usepackage{mathrsfs}
\usepackage{booktabs}
\usepackage[pdftex]{hyperref}
\usepackage{lscape}
\captionsetup{justification=RaggedRight, singlelinecheck=false}
\newcommand{\ra}[1]{\renewcommand{\arraystretch}{#1}}
\newcommand{\argmax}{\text{argmax}}
\newcommand{\Tr}{\text{Tr}}
\newcommand{\z}{\mathscr{Z}}
%\newtheorem{claim}{Claim}

\addtolength{\evensidemargin}{-.5in}
\addtolength{\oddsidemargin}{-.5in}
\addtolength{\textwidth}{1.4in}
\addtolength{\textheight}{1.4in}
\addtolength{\topmargin}{-.5in}

\pagestyle{empty}

\begin{document}


\begin{center}
{\bf \LARGE{Evolution of Information Gathering Strategies}}
\vspace{10pt}
\\ Eleanor Brush
\\ October 31, 2014
\end{center}

\tableofcontents

\section{Abstract}
Birds in flocks, like animals in many types of social groups, can use their peers to learn about the environment and update their opinions about where to move in that environment.  Previous work on starlings has measured the number of other birds an individuals pays attention to and shown that this number of ``neighbors" leads to a social network that is conducive to the whole flock reaching consensus.  It is unclear, however, why an individual would change its behavior, i.e. the number of neighbors it learns from, to achieve this outcome.  Individual birds should optimize how well they learn about the environment. There are (at least) two features of the environment that are important to an individual bird's fitness---the location of a predator and the location of resources---and it can try to learn about these from its peers. In this work, we identify the optimal strategies for birds under selection pressure dictated by these types of information and explore the relationship between the robustness measure and the correlation length of the flock, a measure commonly used to describe how close a flock's structure is to criticality.


\section{Model }
\subsection{Learning dynamics }
An individual's opinion changes with a probability that depends on its social interactions and the strength of the environmental stimulus. Specifically, an individual's opinion $\sigma_i$ can be either $1$ or $-1$ and $\sigma_i$ switches with probability $w_i(\sigma_i)=\frac{1}{2}(-\sigma_i\sum_jA_{ij}\sigma_j )(1-\beta_i\sigma_i)$ where $A_{ij}$ indicates the strength of the connection from $j$ to $i$ and $\beta_i$ indicates the strength of the environmental stimulus perceived by $i$.  The stochastic dynamics of the opinions can be written down with a master equation \cite{Glauber:1963fk}.  The vector $\vec{v}(t)$ of expected values, $q_i(t)=\langle \sigma_i(t)\rangle$, satisfies the following differential equations: \cite{Glauber:1963fk}
\begin{equation*}
\frac{d\vec{v}}{dt}=M\vec{v}+\vec{\beta}^T(\vec{1}-\vec{v}).
\end{equation*}
We simplify this by writing $\frac{d\vec{v}}{dt}=A\vec{v}+\vec{\beta}^T\vec{1}$, where $A=M-B$ and $B$ is a diagonal matrix with the values of $\vec{\beta}$ along the diagonal.

These stochastic dynamics will lead to an equilibrium probability distribution over sets of opinions that agrees with the Boltzmann equilibrium distribution for a system with energy or Hamiltonian given by 
\begin{equation*}
E(\vec{v})=-\vec{v}^TA\vec{v}-\vec{\beta}^T\vec{v}
\end{equation*}  

\subsection{Individual-level fitness }
To find the optimal strategies from the individuals' perspectives, we need a measure of individual-level fitness, which we define as follows. Each individual has a strategy that determines how many other individuals he pays attention to, $n_i=\sum_{j\neq i}{\bf I}(A_{ij}\neq 0)$.  We distribute the individuals randomly in space.  Each individual then pays attention to its $n_i$ nearest neighbors.  One individual is chosen to be a receiver; it and all those individuals within a radius $r$ of the receiver perceive an external signal of strength $\beta$.  We find the expected opinion of each individual $q_i$ after a period of time $T$.  We refer to each such set of random positions and receiver as a signaling event. 

We consider two types of signals: predators and food.  If the signal is information that a predator is present, we assume that whichever individual is least aware of the signal will be predated, i.e. the individual $i$ such that $q_i$ is lowest.  We therefore find, over many signaling events, the probability that an individual will be eaten and an individual's fitness is given by the probability of surviving, i.e. $1$ minus the probability of being predated.  If the signal is information that resources are available, we assume that whichever individual is most aware of the signal will get access to the resource, i.e. the individual $i$ such that $q_i$ is highest. In this case, fitness is proportional to the probability of being the first to reach the resources. 

\subsection{Group-level performance }
We consider two group-level properties that indicate how cohesive the group is.  The first property is $\mathscr{H}_2$ robustness, as used in \cite{Young:2010fk,Young:2013kx}. This is a measure of the robustness of the consensus state in which all birds have the same opinion to noise at equilibrium, as used in (see Appendix Sec. \ref{H2}). The second property is the correlation length of the flock. This is the distance at which the average correlations between birds changes from positive to negative, i.e. the distance over which birds' opinions tend to be positively correlated with each other (see Appendix Sec \ref{corr}).

\subsection{Optimization methods }
To understand what strategies we might expect to find, we are interested in identifying the optimal strategies.  We does this in two ways.  First, using the framework of adaptive dynamics in finite populations, we identify the evolutionarily stable strategy.  This framework assumes that there is a homogeneous population into which a mutant individual tries to invade.  If strategies change on a learning rather than an evolutionary timescale, or if there is just a lot of variation in the population, we might expect individuals to try to optimize their strategies in the context of a heterogeneous population.  This is our second method of optimization. Given a random set of initial strategies over the group, we allow each bird to choose the strategy that would be best given the rest of the strategies being used and repeat this process until the birds reach an equilibrium set of strategies. We repeat this over many initial sets of strategies to find average properties of this optimization process.

\section{Results }

\subsection{ESS strategies }
Surprisingly, even without imposing costs on paying attention to more neighbors, having as many neighbors as possible is usually sub-optimal. The ESS strategies for the number of neighbors to pay attention strongly depends on which selection pressure is being applied. The ESS strategy when selection is due to knowledge of a predator is always higher than the ESS strategy when selection is due to knowledge about resources (Figure \ref{ESS}). When selection is due to predation, the ESS strategy is a non-monotonic function of the the radius of the signal, i.e. how public the signal is (Figure \ref{ESS}). This non-monotonic behavior is due to two opposing effects of paying attention to more neighbors. First, having more neighbors increases the probability that one of them will have the signal and thus the probability that the focal bird will have a good source of information. Second, having more neighbors increases the probability that a focal bird will be trying to learn from individuals who do not have true information about the environment. When few individuals have the true signal (i.e. low radius $r$), the second effect outweighs the first and the ESS strategies are to pay attention to fewer birds than are in the whole flock. However, as more individuals have the true signal (i.e. higher radius $r$), the first effect is more important 

\subsection{Learned strategies }
The equilibrium strategies when the individuals are allowed to learn are similar to the ESS strategies: learned strategies when selection is due to predation are always higher than learned strategies when selection is due to resources (Figure \ref{greedyopt}). Surprisingly, even when the group starts with heterogeneous strategies, they often reach a homogeneous equilibrium. Under selection due to predation, the $\mathscr{H}_2$ robustness and correlation length of the flock tend to decrease, whereas under selection due to resources, $\mathscr{H}_2$ robustness is relatively constant and correlation length tends to increase (Figure \ref{greedyopt}).

\subsection{Relationship between group properties }
Figure \ref{greedyopt} shows that there is a relationship between two group properties: the $\mathscr{H}_2$ norm and the correlation length of the flock.

\section{Discussion}

\section{Figures }
\begin{figure}[ht]
\includegraphics[width=6.83in]{/Users/eleanorbrush/Desktop/ESSneighbors.pdf}
\caption{\label{ESS} The ESS number of neighbors is always higher if selection is due to predation than due to resources.  The ESS number of neighbors when selection is due to predation is a non-monotonic function of how public the signal is. The x-axis indicates the radius of the signal and the y-axis indicates the ESS number of neighbors. Parameters:  $\beta=1$, $T=1$. 
}
\end{figure}

\begin{figure}[ht]
\includegraphics[width=6.83in]{/Users/eleanorbrush/Desktop/greedyoptneighbors.pdf}
\caption{\label{greedyopt} When the birds choose the optimal number of neighbors, given the strategies the rest of the flock are using, they settle on higher strategies when selection is due to predation than when selection is due to resources. Under selection due to predation, the $\mathscr{H}_2$ robustness and correlation length of the flock tend to decrease, whereas under selection due to resources, $\mathscr{H}_2$ robustness is relatively constant and correlation length tends to increase. The upper row shows results from implementing selection based to predation and the lower row shows results from implementing selection based on resources. In each panel, the x-axis represents the number of times the birds are allowed to choose optimal strategies. The first column shows one example of how the birds' strategies change over time. The second column shows, for many initial conditions, how the $\mathscr{H}_2$ robustness changes over time. The third column shows, for many initial conditions, how the correlation length changes over time. Parameters:  $\beta=1$, $r=.1$, $T=1$. 
}
\end{figure}

\newpage
\section{Appendix}
\subsection{H2 Norm \label{H2}} 

\ Let $y$ represent the distance between the vector of opinions and consensus, i.e.
$y=QxQ^T,$
where $Q\in\R^{n-1\times n}$ is such that $Q\vec{1}=0$ , $QQ^T=I_{n-1}$, and $Q^TQ=I_n-\frac{1}{n}\vec{1}\times\vec{1}^T$.   
Then 
\begin{equation}
\dot{y}(t)=-\overline{L}y(t)+Q\xi(t) \notag
\end{equation}
where $\overline{L}=QLQ^T$.  As shown in \cite{Young:2010fk}, if $\Sigma_y(t)=\E[y(t)y(t)^T]$,  $\Sigma_y$ at equilibrium solves
\begin{equation} 0=-\overline{L}\Sigma_y-\Sigma_y\overline{L}^T+D. \label{h2norm}
\end{equation}
\begin{align*}
\mathscr{H}_2&=\sqrt{\Tr(\Sigma_y)}
\\\mathscr{H}_2 \text{ robustness} &=\frac{1}{\mathscr{H}_2}
\end{align*}

\subsection{Calculating correlations \label{corr}}
The following is derived from the work in \cite{Bialek:2013fk}.
\begin{align*}
H(\vec{v})&=-\vec{v}^TA\vec{v}-\vec{\beta}^T\vec{v}
%\\&=\vec{v}A\vec{v}-\vec{\beta}^T\vec{v}+\vec{\beta}^T\vec{1}
\end{align*}
Define $V=\sum_kv_k/N$ and $\vec{y}=(\vec{v}-V\vec{1})/V$.  Then $\vec{v}=V(\vec{1}+\vec{y})$.  
\begin{align*}
H(\vec{y},V)&=-V^2(\vec{1}^T+\vec{y}^T)A(\vec{1}+\vec{y})-\vec{\beta}^TV(\vec{1}+\vec{y})
\\&=-V^2(\vec{1}^T+\vec{y}^T)A\vec{y}-V\vec{\beta}^T\vec{y}-V\vec{\beta}^T\vec{1} \text{ since $A\vec{1}=\vec{0}$}
\\&=-V^2\vec{y}^TA\vec{y}+(-V^2\vec{1}^TA-V\vec{\beta}^T)\vec{y}-V\vec{\beta}^T\vec{1}
\end{align*}
Define $Q\in M_{N-1,N}$ that rotates an $N$-vector away from the consensus vector and define $\vec{z}=Q\vec{y}$ so that $\vec{y}=Q^T\vec{z}$ since $\sum_ky_k=0$.
\begin{align*}
H(\vec{z},V)&=-V^2\vec{z}^TQAQ^T\vec{z}+(-V^2\vec{1}^TAQ^T-V\vec{\beta}^TQ^T)\vec{z}-V\vec{\beta}^T\vec{1}
\\&=V^2\vec{z}^TP\vec{z}+(-V^2\vec{\sigma}_1^T-V\vec{\sigma}_2^T)\vec{z}-V\vec{\beta}^T\vec{1}
\\ \Rightarrow P(\vec{z},V)&=\frac{1}{\z}\exp\left(-V^2\vec{z}^TP\vec{z}+(V^2\vec{\sigma}_1^T+V\vec{\sigma}_2^T)\vec{z}+V\vec{\beta}^T\vec{1}\right)
\end{align*}
where $P=-QAQ^T$, $\vec{\sigma}_1=QA^T\vec{1}$, and $\vec{\sigma}_2=Q\vec{\beta}$.


\begin{fact}
\begin{align*}
\int\exp\left(-\frac{1}{2}\vec{z}^TP\vec{z}+\vec{\sigma}^T\vec{z}\right)d^{N-1}z&=\sqrt{\frac{(2\pi)^{N-1}}{\det P}}\exp\left(-\frac{1}{2}\vec{\sigma}^TP^{-1}\vec{\sigma}\right)
\\\Rightarrow \int\exp\left(-\vec{z}^TP\vec{z}+\vec{\sigma}^T\vec{z}\right)d^{N-1}z&=\sqrt{\frac{(2\pi)^{N-1}}{2^{N-1}\det P}}\exp\left(-\frac{1}{4}\vec{\sigma}^TP^{-1}\vec{\sigma}\right)
\\&=\sqrt{\pi^{N-1}}\sqrt{\frac{1}{\det P}}\exp\left(-\frac{1}{4}\vec{\sigma}^TP^{-1}\vec{\sigma}\right)
\end{align*}
\end{fact}

\begin{align*}
\z&=\int_{-\infty}^\infty\int_{\R^{N-1}}\exp\left(V^2\vec{z}^TP\vec{z}+(V^2\vec{\sigma}_1^T+V\vec{\sigma}_2^T)\vec{z}+V\vec{\beta}^T\vec{1}\right)d^NzdV
\\&=\int_{-\infty}^\infty \exp(V\vec{\beta}^T\vec{1})\int_{\R^{N-1}}\exp\left(V^2\vec{z}^TP\vec{z}+(V^2\vec{\sigma}_1^T+V\vec{\sigma}_2^T)\vec{z}\right)d^NzdV
\\&=\sqrt{\pi^{N-1}}\int_{-\infty}^\infty\exp(V\vec{\beta}^T\vec{1})\times\left(\sqrt{\frac{1}{V^{2(N-1)}\det{P}}}\right)\exp\left(-\frac{1}{4}\vec{\sigma}_1^TP^{-1}\vec{\sigma}_1\right)\exp\left(-\frac{1}{4V}\vec{\sigma}_2^TP^{-1}\vec{\sigma}_2\right)dV
\\&=\sqrt{\pi^{N-1}}\sqrt{\frac{1}{\det PP}}\exp\left(-\frac{1}{4}\vec{\sigma}_1^TP^{-1}\vec{\sigma}_1\right)\int_{-\infty}^\infty\exp(V\vec{\beta}^T\vec{1})\times\left(\sqrt{\frac{1}{V^{2(N-1)}}}\right)\exp\left(-\frac{1}{4V}\vec{\sigma}_2^TP^{-1}\vec{\sigma}_2\right)dV
\\&=\sqrt{\pi^{N-1}}\sqrt{\frac{1}{\det P}}\exp\left(-\frac{1}{4}\vec{\sigma}_1^TP^{-1}\vec{\sigma}_1\right)\int_{-\infty}^\infty\frac{1}{V^{(N-1)}}\exp\left(V\vec{\beta}^T\vec{1}-\frac{1}{4V}\vec{\sigma}_2^TP^{-1}\vec{\sigma}_2\right)dV
\\&=\sqrt{\pi^{N-1}}\sqrt{\frac{1}{\Pi_a\lambda_a}}\exp\left(-\frac{1}{4}\vec{\sigma}_1^TP^{-1}\vec{\sigma}_1\right)\int_{-\infty}^\infty\frac{1}{V^{(N-1)}}\exp\left(V\vec{\beta}^T\vec{1}-\frac{1}{4V}\vec{\sigma}_2^TP^{-1}\vec{\sigma}_2\right)dV 
\\&\text{ where $\lambda_a$ are the eigenvalues of $P$}
\\ \Rightarrow -\log(\z)&=\text{constants}+\frac{1}{2}\sum_a\log(\lambda_a)+\frac{1}{4}\vec{\sigma}_1^TP^{-1}\vec{\sigma_1}-\log(f(P))
\end{align*}

where $f(P)=\int_{-\infty}^\infty\frac{1}{V^{(N-1)}}\exp\left(V\vec{\beta}^T\vec{1}-\frac{1}{4V}\vec{\sigma}_2^TP^{-1}\vec{\sigma}_2\right)dV $.

\begin{align*}
\Rightarrow\frac{\partial -\log(\z)}{\partial P_{ij}}&=\frac{1}{2}\sum_a\frac{w_iw_j}{\lambda_a}+\frac{1}{4}\sum_{kl}\sigma_{1k}P^{-1}_{ki}P^{-1}_{jl}\sigma_{1l}-\frac{\partial \log f}{\partial P_{ij}}
\\\Rightarrow\frac{\partial -\log(\z)}{\partial P_{ij}}&=\frac{1}{2}\sum_a\frac{w_iw_j}{\lambda_a}+\frac{1}{4}\sum_{kl}\sigma_{1k}P^{-1}{ki}P^{-1}_{jl}\sigma_{1l}-\frac{1}{f}\frac{\partial f}{\partial P_{ij}}
\end{align*}

\begin{align*}
\frac{\partial f}{\partial P_{ij}}&=\int_{-\infty}^\infty\frac{1}{V^{(N-1)}}\exp\left(V\vec{\beta}^T\vec{1}-\frac{1}{4V}\vec{\sigma}_2^TP^{-1}\vec{\sigma}_2\right)\left(-\frac{1}{4V}\sum_{kl}\sigma_{2k}P^{-1}_{ki}P^{-1}_{jl}\sigma_{2l}\right)dV
\\&=\left(-\frac{1}{4}\sum_{kl}\sigma_{2k}P^{-1}_{ki}P^{-1}_{jl}\sigma_{2l}\right)\int_{-\infty}^\infty\frac{1}{V^{(N)}}\exp\left(V\vec{\beta}^T\vec{1}-\frac{1}{4V}\vec{\sigma}_2^TP^{-1}\vec{\sigma}_2\right)dV
\end{align*}

\small{
\begin{align*}
\Rightarrow\langle z_iz_j\rangle &=\frac{1}{2}\sum_a\frac{w_iw_j}{\lambda_a}+\frac{1}{4}\sum_{kl}\sigma_{1k}P^{-1}{ki}P^{-1}_{jl}\sigma_{1l}+\left(\frac{1}{4}\sum_{kl}\sigma_{2k}P^{-1}_{ki}P^{-1}_{jl}\sigma_{2l}\right)\frac{\int_{-\infty}^\infty\frac{1}{V^{(N)}}\exp\left(V\vec{\beta}^T\vec{1}-\frac{1}{4V}\vec{\sigma}_2^TP^{-1}\vec{\sigma}_2\right)dV}{\int_{-\infty}^\infty\frac{1}{V^{(N-1)}}\exp\left(V\vec{\beta}^T\vec{1}-\frac{1}{4V}\vec{\sigma}_2^TP^{-1}\vec{\sigma}_2\right)dV}
\end{align*}}


\subsection{Equivalence of dynamics and Hamiltonian }
\begin{claim}
If $H(\vec{v})=-\vec{v}^TA\vec{v}-\vec{\beta}^T\vec{v}$ and $\frac{d\vec{v}}{dt}=A\vec{v}+\vec{\beta}$ then $\nabla H\cdot \frac{d\vec{v}}{dt}\leq 0$.
\end{claim}
\begin{pf}
\begin{align*}
H(\vec{v})&=-\sum_{ij}A_{ij}v_iv_j-\sum_i\beta_iv_i
\\\Rightarrow\frac{\partial H}{\partial v_i}&=-\sum_{j\neq i}A_{ij}v_j-\sum_{j\neq i}A_{ji}v_j-2A_{ii}v_i-\beta_i
\\&=-\sum_j(A_{ij}+A^T_{ij})v_j-\beta_i
\\ \Rightarrow \nabla H&=-(A+A^T)\vec{v}-\vec{\beta}
\\ \Rightarrow \nabla H\cdot \frac{d\vec{v}}{dt}&=-\vec{v}^T(A+A^T)A\vec{v}-\vec{v}^T(A+A^T)\vec{\beta}-\vec{\beta}^TA\vec{v}-\vec{\beta}^T\vec{\beta}
\\&=-\vec{v}^TA^2\vec{v}-\vec{v}^TA^TA\vec{v}-\vec{v}^TA\vec{\beta}-2\vec{v}^TA^T\vec{\beta}-\vec{\beta}^T\vec{\beta}
\end{align*}
If $A$ is negative semi-definite, then $A^2$ is positive semi-definite and $A^TA$ is positive semi-definite.  (Consider that $\vec{v}^TA^TA\vec{v}=(A\vec{v})^T(A\vec{v})$ so that $\vec{v}A^TA\vec{v}=0$ if $A\vec{v}=0$ and $\vec{v}^TA^TA\vec{v}<0$ otherwise.) Therefore, $-\vec{v}^TA^2\vec{v}\leq 0$ and $-\vec{v}^TA^TA\vec{v}\leq 0$. At equilibrium, $\vec{v}$ will be ``close" to $\vec{\beta}$ so that $$-\vec{v}^TA\vec{\beta}-2\vec{v}^TA^T\vec{\beta}\sim-\vec{\beta}^TA\vec{\beta}$$
\end{pf}

\subsection{Numerically evaluating dynamics }
\begin{claim}
Let $\Phi(t)$ be the fundamental matrix solution to the homogeneous differential equations $\frac{d\vec{x}}{dt}=A\vec{x}$ so that $\Phi(t)=V\Lambda(t)$ where the columns of $V$ are the eigenvectors of $A$ and $$\Lambda(t)=\text{diag}(e^{\lambda_1 t},\dots,e^{\lambda_n t})$$ where $\{\lambda_i\}$ are the eigenvalues of $A$.  (Note that if $\lambda_i=0$ then $e^{\lambda _i t}=1$ for all $t$.) Let $V^\dagger$ be the pseudoinverse of $V$.  Let $\vec{c}=V^\dagger\vec{X_0}$.  Let $$G(t)=\text{diag}(1/\lambda_1(1-e^{-\lambda_1t}),\dots,1/\lambda_n(1-e^{-\lambda_nt}))V^\dagger\vec{\beta}.$$
(If $\lambda_i=0$ then the corresponding entry of the first part of $G(t)$ will be given by $t$ rather than the form above.)  THEN if we let $x(t)=\Phi(t)G(t)+\Phi(t)\vec{c}$, $X(t)$ solves the inhomogeneous equations $\frac{dx}{dt}=A\vec{x}+\vec{\beta}$ with initial conditions $x(0)=\vec{X_0}$.
\end{claim}

\begin{pf}
It is clear that
\begin{align*}
\frac{d\Phi\vec{c}}{dt}&=A\Phi(t)\vec{c} \text{ and } \Phi(0)\vec{c}=\vec{X_0}.
\end{align*}
Now, 
\begin{align*}
\Phi(t)G(t)&=V
\left(\begin{array}{ccccc}
e^{\lambda_1 t}  &\dots &0
\\ \vdots & \ddots &\vdots
\\ 0 & \dots & e^{\lambda_nt}
 \end{array}\right)
 \left(\begin{array}{ccccc}
\frac{1}{\lambda_1}(1-e^{\lambda_1 t})  &\dots &0
\\ \vdots & \ddots &\vdots
\\ 0 & \dots &\frac{1}{\lambda_n} (1-e^{\lambda_nt})
 \end{array}\right)
 V^\dagger\vec{\beta}
 \\&=V
 \left(\begin{array}{ccccc}
\frac{1}{\lambda_1}(e^{\lambda_1 t}-1)  &\dots &0
\\ \vdots & \ddots &\vdots
\\ 0 & \dots &\frac{1}{\lambda_n} (e^{\lambda_nt}-1)
 \end{array}\right)
 V^\dagger\vec{\beta}
 \\ \Rightarrow \frac{d\Phi(t)G(t)}{dt}&=V
 \left(\begin{array}{ccccc}
e^{\lambda_1 t}  &\dots &0
\\ \vdots & \ddots &\vdots
\\ 0 & \dots &e^{\lambda_nt}
 \end{array}\right)
 \\&=V
 \left(\begin{array}{ccccc}
e^{\lambda_1 t}  &\dots &0
\\ \vdots & \ddots &\vdots
\\ 0 & \dots &e^{\lambda_nt}
 \end{array}\right)
 V^\dagger\vec{\beta} -VV^\dagger\vec{\beta}+\vec{\beta}
 \\&=V
 \left(\begin{array}{ccccc}
e^{\lambda_1 t}-1  &\dots &0
\\ \vdots & \ddots &\vdots
\\ 0 & \dots &e^{\lambda_nt}-1
 \end{array}\right)
 V^\dagger\vec{\beta} +\vec{\beta}
  \\&=V
 \left(\begin{array}{ccccc}
\lambda_1e^{\lambda_1t}\cdot \frac{1}{\lambda_1}(1-e^{-\lambda_1 t})  &\dots &0
\\ \vdots & \ddots &\vdots
\\ 0 & \dots &\lambda_ne^{\lambda_nt}\cdot\frac{1}{\lambda_n}(1-e^{-\lambda_nt})
 \end{array}\right)
 V^\dagger\vec{\beta} +\vec{\beta}
  \\&=AV\Lambda(t)
 \left(\begin{array}{ccccc}
 \frac{1}{\lambda_1}(1-e^{-\lambda_1 t})  &\dots &0
\\ \vdots & \ddots &\vdots
\\ 0 & \dots &\frac{1}{\lambda_n}(1-e^{-\lambda_nt})
 \end{array}\right)
 V^\dagger\vec{\beta} +\vec{\beta}
 \\&=AV\Lambda(t)G(t)+\vec{\beta}
 \\&=A\Phi(t)G(t)+\vec{\beta}
\end{align*}
Therefore $\frac{dx}{dt}=A\Phi(t)G(t)+A\Phi(t)\vec{c}+\vec{\beta}=Ax(t)+\vec{\beta}$ and since $G(0)=\vec{0}$, $x(0)=\vec{X_0}$.
\end{pf}

\nocite{*}
\bibliographystyle{plain}
\bibliography{info_evo}

\end{document}


