\documentclass{article}
%\documentclass[aps, twocolumn, tightenlines,amscd,amsmath,amssymb,verbatim]{revtex4}
\usepackage{latexsym}
\usepackage{amssymb,amsmath}
\usepackage{custom2}
\usepackage{graphicx} % for figures
\usepackage{epstopdf} % so can use EPS or PDF figures
%\usepackage{subfig}
\usepackage{caption}
\usepackage{subcaption}
\usepackage{url}
\usepackage{amssymb,amsfonts}
\usepackage[all,arc]{xy}
\usepackage{enumerate}
\usepackage{mathrsfs}
\usepackage{booktabs}
\usepackage[pdftex]{hyperref}
\usepackage{lscape}
\captionsetup{justification=RaggedRight, singlelinecheck=false}
\newcommand{\ra}[1]{\renewcommand{\arraystretch}{#1}}
\newcommand{\argmax}{\text{argmax}}
\newcommand{\Tr}{\text{Tr}}
\newcommand{\z}{\mathscr{Z}}
%\newtheorem{claim}{Claim}

\addtolength{\evensidemargin}{-.5in}
\addtolength{\oddsidemargin}{-.5in}
\addtolength{\textwidth}{1.4in}
\addtolength{\textheight}{1.4in}
\addtolength{\topmargin}{-.5in}

\pagestyle{empty}

\begin{document}

%
%\begin{center}
%{\bf \LARGE{Evolution of Information Gathering Strategies}}
%\vspace{10pt}
%\\ Eleanor Brush
%\\ October 31, 2014
%\end{center}

\tableofcontents

\section{Abstract}
Birds in flocks, like animals in many types of social groups, can use their peers to learn about the environment and update their opinions about where to move in that environment.  Previous work on starlings has measured the number of other birds an individuals pays attention to and shown that this number of ``neighbors" leads to a social network that is conducive to the whole flock reaching consensus.  It is unclear, however, why an individual would change its behavior, i.e. the number of neighbors it learns from, to achieve this outcome.  Individual birds should optimize how well they learn about the environment. There are (at least) two features of the environment that are important to an individual bird's fitness---the location of a predator and the location of resources---and it can try to learn about these from its peers. In this work, we identify the optimal strategies for birds under selection pressure dictated by these types of information and explore the relationship between the robustness measure and the correlation length of the flock, a measure commonly used to describe how close a flock's structure is to criticality.


\section{Model }
\subsection{Learning dynamics }
An individual's opinion changes with a probability that depends on its social interactions and the strength of the environmental stimulus. Let $A$ be the adjacency matrix such that 

\begin{equation}
A_{ij}=\left\{\begin{array}{l l l}
1 & \text{ if $i$ pays attention to $j$ }
\\0 & \text{ otherwise }
\end{array}\right.
\end{equation}

We assume that having more neighbors means paying less attention to each so we normalize $A$ to find $\bar{A}$:
\begin{equation}
\bar{A}_{ij}=\frac{A_{ij}}{\sum_kA_{ik}}.
\end{equation}

If each bird averages the differences between its opinion and each of its neighbors' to update its opinion, the opinion dynamics will be described by 
\begin{equation}
\dot{x}=Lx
\end{equation}
where $L=\bar{A}-I$. In the absence of the environmental signal, there is no trend in the dynamics. We introduce an environmental signal by changing the dynamics such that informed birds change their opinions to be closer to $1$. Having an opinion closer to $1$ can be interpreted as having more information about the environment and having a lower expected value can be interpreted as having less information. Specifically, we incorporate this environmental forcing with a diagonal matrix $B$ where $B_{ii}=b$ if $i$ perceives the environmental signal and $B_{ii}=0$ otherwise. The dynamics then become
\begin{equation}
\dot{x}=Lx+B(\vec{1}-x).
\end{equation}
If we let $L_\text{s}=L-B$, then we can simplify this by writing
\begin{equation}
\dot{x}=L_\text{s}x+\vec{B}
\end{equation}
where $\vec{B}=B\vec{1}$ indicates which birds perceive the environmental signal.



\subsection{Individual-level fitness }
To find the optimal strategies from the individuals' perspectives, we need a measure of individual-level fitness, which we define as follows. Each individual has a strategy that determines how many other individuals he pays attention to, $n_i=\sum_{j\neq i}{\bf I}(A_{ij}\neq 0)$.  We distribute the individuals randomly in space.  Each individual then pays attention to its $n_i$ nearest neighbors.  One individual is chosen to be a receiver; it and all those individuals within a radius $r$ of the receiver perceive an external signal of strength $\beta$.  We find the expected opinion of each individual $q_i$ after a period of time $T$.  We refer to each such set of random positions and receiver as a signaling event. 

We consider two types of signals: predators and food.  If the signal is information that a predator is present, we assume that whichever individual is least aware of the signal will be predated, i.e. the individual $i$ such that $q_i$ is lowest.  We therefore find, over many signaling events, the probability that an individual will be eaten and an individual's fitness is given by the probability of surviving, i.e. $1$ minus the probability of being predated.  If the signal is information that resources are available, we assume that whichever individual is most aware of the signal will get access to the resource, i.e. the individual $i$ such that $q_i$ is highest. In this case, fitness is proportional to the probability of being the first to reach the resources. 

\subsection{Group-level performance }
%%There are two ways to describe equilibrium properties
We consider two group-level properties that indicate how cohesive the group is.  The first property is $\mathscr{H}_2$ robustness, as used in \cite{Young:2010fk,Young:2013kx}. This is a measure of the robustness of the consensus state in which all birds have the same opinion to noise at equilibrium, as used in (see Appendix Section XXX). The second property is the correlation length of the flock. This is the distance at which the average correlations between birds changes from positive to negative, i.e. the distance over which birds' opinions tend to be positively correlated with each other (see Appendix Section XXX).

\subsection{Optimization methods }
To understand what strategies we might expect to find, we are interested in identifying the optimal strategies.  We does this in two ways.  First, using the framework of adaptive dynamics in finite populations, we identify the evolutionarily stable strategy.  This framework assumes that there is a homogeneous population into which a mutant individual tries to invade.  If strategies change on a learning rather than an evolutionary timescale, or if there is just a lot of variation in the population, we might expect individuals to try to optimize their strategies in the context of a heterogeneous population.  This is our second method of optimization. Given a random set of initial strategies over the group, we allow each bird to choose the strategy that would be best given the rest of the strategies being used and repeat this process until the birds reach an equilibrium set of strategies. We repeat this over many initial sets of strategies to find average properties of this optimization process.

\section{Results }

\subsection{ESS strategies }
Surprisingly, even without imposing costs on paying attention to more neighbors, paying attention to the whole flock is not usually the best strategy. The ESS strategy of the number of neighbors to pay attention strongly depends on which selection pressure is being applied.  If a focal bird pays attention to many others, its opinion will be similar to theirs. This will prevent it from being the most informed and hence the first to locate resources. Thus, the uninvadable strategy is to pay attention to few neighbors when selection is due to knowledge about resources. On the other hand, having many neighbors will prevent the focal bird from being the least informed and thus the most likely to be eaten.  Thus, the uninvadable strategy is to pay attention to many neighbors when selection is due to knowledge about predators. Figure \ref{ESS}a shows the ESS strategy when selection is due to knowledge about resources is always lower than the ESS strategy when selection is due to knowledge about predators (Figure \ref{ESS}a).

The ESS strategies when selection is due to knowledge about predators show two interesting features. First, there are many parameters for which there are multiple ESS strategies. It might seem that if a strategy outperforms all others when it is the resident strategy, it should also outperform other strategies as an invader. However, when there are multiple ESS strategies, if we take any two of those ESS strategies, each is uninvadable by the other. This indicates how important the social environment set by the resident strategy is. The pairwise invisibility plot in Figure \ref{ESS}b shows that low strategies are always costly and that high strategies are only costly when the resident strategy is high. 

If a bird has the environmental signal, it will not be predated, regardless of the number of neighbors it has. The differences in performance only appear when comparing two birds, neither of whom have the signal. To understand the costs of having too few or too many neighbors, we compare the performance of two focal birds, a resident and an invader, neither of whom has the signal.  There are two main cases: there can be a single informed bird or multiple informed birds. We further break down the first case depending on which of these two focal birds pays attention to the informed bird: neither, the resident, the invader, or both. For each of these cases, we find the probability that the invader is predated. An invader with one fewer neighbor than the resident always has probability of being predated greater than chance ($1/N$) because the resident often has an informed neighbor while the invader how no or fewer informed neighbors and thus a less informed opinion (Figure \ref{ESS}c). The total probability of being predated stays more or less constant regardless of the resident strategy, with a slight decrease for high resident strategies (Figure \ref{ESS}b and c). An invader with one more neighbor than the resident is only more likely than chance to be predated when the resident strategy is high enough  (Figure \ref{ESS}d). This invader performs most poorly when both it and the focal resident pay attention to the informed bird(s). In this case, an invader with one more neighbor has more naive neighbors and thus a less informed opinion. This is increasingly problematic as the resident strategy increases and both the focal birds are likely to pay attention to the informed bird(s).

These opposing forces caused by the costs of not paying attention to informed neighbors  and the costs of paying attention to more uninformed neighbors  also explains the second feature of the ESS strategies: they are non-monotonic with respect to the the radius of the signal, i.e. how public the signal is (Figure \ref{ESS}a). Figure \ref{ESS}a shows that there are essentially three categories of radii: When the radius is low, the signal is private, with only a few birds having true information. When the radius is intermediate, the signal is public, with many birds having true information. When the radius is high, the signal is universal, with essentially all birds in the flock having true information. When the signal is private, paying attention to too few birds is costly because with more neighbors it is more likely to have a truly informed neighbor, but paying attention to too many uninformed birds can also be costly. This leads to a high but not maximal ESS strategy. When the signal is public, the costs of paying attention to uninformed birds are reduced and the ESS strategy increases to become as high as possible. When the signal is universal, not paying attention to informed birds does not impose a cost. Additionally, in a group in which the residents pay attention to all of their neighbors, any mutant strategy can get just as much information by paying attention to fewer neighbors so the maximal strategy is no longer an ESS (Figure \ref{ESS}a).

Another way to explain the non-monotonic ESS number of neighbors is to consider the correlation length of a homogeneous flock as a function of the radius of the signal. If a flock is all using the same strategy, its average correlation length depends much more strongly on the radius of the signal than the strategy being used (Figure \ref{corrESS}a). In particular, the correlation length first rises and then falls as the radius is increased. The rise and fall of the correlation length correspond to the rise and fall of the ESS number of neighbors and if the we plot the ESS number of neighbors for various radii of the signal against the average correlation length of a homogeneous flock for that radii, we find a positive relationship (Figure \ref{corrESS}b). This is because when correlations in information die off very quickly (i.e. a low correlation length), having more neighbors is costly because birds even slightly away from the informed birds will have poor information. When there are strong correlations in information across the flock, on the other hand, these costs are reduced because every bird in the flock has on average higher quality information.

\subsection{Learned strategies }
The equilibrium strategies when the individuals are allowed to learn are similar to the ESS strategies: learned strategies when selection is due to predation are always higher than learned strategies when selection is due to resources (Figure \ref{greedyopt}). Surprisingly, even when the group starts with heterogeneous strategies, they often reach a homogeneous equilibrium. Under selection due to predation, the $\mathscr{H}_2$ robustness and correlation length of the flock tend to decrease, whereas under selection due to resources, $\mathscr{H}_2$ robustness is relatively constant and correlation length tends to increase (Figure \ref{greedyopt}).

\subsection{Relationship between group properties }
Figure \ref{greedyopt} shows that there is a relationship between two group properties: the $\mathscr{H}_2$ norm and the correlation length of the flock.

\section{Discussion}
explore vs. exploit 
risk aversion vs. risk tolerance

expect single ESS strategies but social information depends on the social environment set by the resident strategy
expect ESS strategies to decrease as the signal becomes more public, but we find the opposite

\newpage
\section{Appendix}
%%% APPENDIX NOW DISCUSSES H2 WITHOUT FORCING (AS IN GEORGE'S PAPER), H2 WITH FORCING, WHICH GIVES CORRELATIONS BUT NOT FLUCTUATIONS AROUND AVERAGE OPINION, STAT MECH CORRELATIONS WITHOUT FORCING AND STAT MECH CORRELATION WITH FORCING, BOTH OF WHICH GIVE FLUCTUATIONS AROUND AVERAGE OPINION. SO H2 WITH FORCING IS STILL INCOMPARABLE TO ANYTHING ELSE SINCE ITS CORRELATIONS NOT FLUCTUATIONS AROUND AVERAGE OPINIONS. IT'S OKAY FOR NOW.
\subsection{Linear algebra background }

\begin{theorem} \label{lyapeq_theorem}
Given any positive symmetric matrix $Q>0$ there exists a unique positive symmetric matrix $\Sigma>0$ satisfyying $A\Sigma+\Sigma A^T=Q$ if and only if the system $\dot{x}=Ax$ is globally asymptotically stable.
\end{theorem}

\begin{claim} \label{lyapeq_derivation}
Consider the dynamical system,
\begin{equation}
\dot{y}=Ay+v+Q\xi  \label{tofindsigma}
\end{equation}
with $E[\xi(t)]=0$, $E[\xi(t)\xi(s)^T]=\frac{1}{2}\delta(t-s)I$, and $E[y(0)\xi^T(t)]=0$. If all eigenvalues of $A$ are negative and if $\Sigma_t=E[y(t)y(t)^T]$ and $\Sigma=\lim_{t\to\infty}\Sigma_t$, then $\Sigma$ satisfies the Lyapunov equation
\begin{equation*}
A\Sigma+\Sigma A^T=A^{-1}vv^T+vv^T(A^{-1})^T-QQ^T.
\end{equation*}
Further, if $C_{ij}=\lim_{t\to\infty} E[(y_i(t)-E(y_i(t)))(y_j(t)-E(y_j(t)))]$ then 
\begin{equation*}
AC+CA^T=-QQ^T
\end{equation*}
\end{claim}

\begin{pf}
\begin{align*}
\Sigma_t&=E[y(t)y(t)^T]
\\ \Rightarrow \dot{\Sigma}&=E[\dot{y}(t)y(t)^T+y(t)\dot{y}(t)^T]
\\&=E[(Ay+v+Q\xi)y^T+y(y^TA^T+v^T+\xi^TQ^T)]
\\&=AE[yy^T]+E[yy^T]A^T+vE[y^T]+E[y]v^T+QE[\xi y^T]+E[y\xi^T]Q^T
\\&=A\Sigma_t+\Sigma_tA^T+vE[y^T]+E[y]v^T+QE[\xi y^T]+E[y\xi^T]Q^T.
\end{align*}
Note that we can write a solution to (\ref{tofindsigma}) as 
\begin{equation}
y(t)=\exp(At)y(0)+\int_0^t\exp(A(t-s))vds+\int_0^t\exp(A(t-s))Q\xi(s)ds. \label{sde_solution}
\end{equation}
Eq. (\ref{sde_solution}) allows us to write 
\begin{align*}
E[y(t)]&=\exp(At)y(0)+\int_0^t\exp(A(t-s))vds
\\&=\exp(At)y(0)-A^{-1}\exp(A(t-s))v|_{s=0}^{s=t} \text{ which exists because $A$ has no $0$ eigenvalues }
\\&=\exp(At)y(0)-A^{-1}(I-\exp(At))v.
\end{align*}
Note that if we take the limit as $t$ goes to infinity, 
$$\lim_{t\to\infty}E[(y(t))]=\lim_{t\to\infty}\exp(At)y(0)-A^{-1}(I-\exp(At))v=-A^{-1}v $$ 
since all eigenvalues of $A$ are negative.  Eq. (\ref{sde_solution}) also allows us to write
\begin{align*}
E[y \xi^T]&=\exp(At)y(0)E[\xi(t)^T]+\int_0^t\exp(A(t-s))BE[\xi(t)^T]ds+\int_0^t\exp(A(t-s))QE[\xi(s)\xi(t)]ds
\\&=0+0+\frac{1}{2}Q
\\ \Rightarrow \dot{\Sigma} &= A\Sigma_t+\Sigma_tA^T+vE[y(t)^T]+E[y(t)]v^T+\frac{1}{2}QQ^T+\frac{1}{2}QQ^T
\\ &= A\Sigma_t+\Sigma_tA^T+vE[y(t)^T]+E[y(t)]v^T+QQ^T
\\ \Rightarrow \lim_{t\to\infty} \dot{\Sigma}&=A\Sigma+\Sigma A-vv^T(A^{-1})^T-A^{-1}vv^T+QQ^T
\\ \Rightarrow A\Sigma+\Sigma A&=vv^T(A^{-1})^T+A^{-1}vv^T-QQ^T \text{ since $\dot{\Sigma}=0$ at equilibrium}.
\end{align*}

Further, by definition of $C$, 
\begin{align*}
C&=\Sigma-\lim_{t\to\infty}E[y(t)]E[y(t)^T]
\\&=\Sigma-(-A^{-1}v)(-A^{-1}v)^T
\\&=\Sigma-A^{-1}vv^T(A^{-1})^T
\\\Rightarrow AC+CA^T&=A\Sigma+\Sigma A^T-AA^{-1}vv^T(A^{-1})^T
-A^{-1}vv^T(A^{-1})^TA^T
\\&=A\Sigma+\Sigma A^T-vv^T(A^{-1})^T-A^{-1}vv^T
\\&=-QQ^T.
\end{align*}
\end{pf}


\subsection{Opinion dynamics with noise }
In our model, each bird in the flock as an opinion $x_i$, giving an opinion vector $x$. The $\mathscr{H}_2$ robustness of the opinion dynamics is defined by the inverse of the expected distance at equilibrium from the opinion vector to a consensus state in which all birds have the same opinion, i.e. $x_i=\alpha$ for all $i$ and for some $\alpha\in\R$. That is, the dynamics are robust if, at equilibrium, the opinions are close to being in consensus. We can find this distance by rotating the opinion vector into the space orthogonal to the consensus vector, $\vec{1}$. We define a matrix $Q\in\R^{N-1}\times\R^N$ such that each row of $Q$ is orthogonal to $\vec{1}$ and the rows are pairwise orthogonal. Young et al. show in \cite{Young:2010fk} show that this equivalent to finding $Q$ such that $Q\vec{1}=\vec{0}$, $QQ^T=I_{N-1}$, and $Q^TQ=I_N-\frac{1}{N}\vec{1}_N\vec{1}_N^T$.  If $y=Qx$, then $||y||=y^Ty$ describes how far away from consensus $x$ is. Specifically, robustness as in \cite{Young:2010fk} is given by 
\begin{equation}
r=\lim_{t\to\infty}\frac{1}{E[||y||]}=\lim_{t\to\infty}\frac{1}{\Tr E[yy^T]}. \label{H2_robustness}
\end{equation}
We also define a full rotation matrix $R\in\R^N\times\R^N$ such that $R=\left(\begin{array}{cc} Q \\ \frac{1}{\sqrt{N}}\vec{1}^T \end{array}\right)$.

Young et al. \cite{Young:2010fk} model opinion dynamics by assuming that each bird averages the differences between its opinion and each of the opinions of its neighbors and changes its opinion to minimize this difference and that each bird's opinion is subject to independent noise of unit intensity. The equations describing these dynamics can be written as 
\begin{equation}
\dot{x_i}=\frac{\sum_{\text{neighbors $j$ of $i$ } }(x_j-x_i)}{K_i} +\xi_i=\sum_j\bar{A}_{ij}x_j-x_i+\xi_i=\sum_jL_{ij}x_j+\xi_i.
\end{equation}
In vector notation this becomes
\begin{equation}
\dot{x}=Lx+\xi.
\end{equation}
The equilibrium of these dynamics is a consensus state in which all birds have the some opinion, $x=\alpha\vec{1}$ for some $\alpha\in\R$. In the absence of any environmental signal, the consensus opinion can be any value $\alpha$. In the presence of an environmental signal, some birds in the flock perceive the signal and change their opinions to be closer to $1$. The environmental forcing is described by a diagonal matrix $B$ where $B_{ii}=b$ if $i$ perceives the environmental signal and $B_{ii}=0$ otherwise and the dynamics then become
\begin{equation}
\dot{x}=Lx+B(\vec{1}-x)=L_\text{s}x+v.
\end{equation}
where $L_\text{s}=L-B$ and $v=B\vec{1}$. If $b=0$, then $L_\text{s}=L$ and $v=\vec{0}$, so the case with the signal reduces to the case without the signal. 


If $z=Rx$, then the first $N-1$ coordinates of $z$ are $y=Qx$ and the last is $\sum_ix_i/\sqrt{N}$. The dynamics of $z$ follow
\begin{align*}
\dot{z}&=R\dot{x}
\\&=RL_\text{s}x+Rv+R\xi.
\end{align*}
Since the rows of $R$ are orthonormal, $RR^T=I$. Further, since the rows of $R$ are orthonormal, $R$ is invertible, which implies that $R^T=R^{-1}$.  Then we can write
\begin{align*}
\dot{z}&=RL_\text{s}R^Tz+Rv+R\xi
\\&=\tilde{L}_\text{s}z+\tilde{v}+R\xi 
\end{align*}
where $\tilde{L}_\text{s}=RL_\text{s}R^T$ and $\tilde{v}=Rv=RB\vec{1}$. If $b\neq 0$, $L_\text{s}$ and therefore $\tilde{L}_\text{s}$ have all negative eigenvalues and $z=(0, 0, \dots 0, \sqrt{N})^T$ is a globally asymptotically stable equilibrium. Therefore, by Claim \ref{lyapeq_derivation} and Theorem \ref{lyapeq_theorem}, $\lim_{t\to\infty}E[z(t)z(t)^T]=\Sigma_z$ where $\tilde{L}_\text{s}\Sigma+\Sigma\tilde{L}_\text{s}^T=-RR^T$. We can then find $\Sigma_y=\lim_{t\to\infty}E[y(t)y(t)^T]$
by removing the last row and column of $\Sigma_z$.

If $b=0$, $L$ has a $0$ eigenvalue corresponding to the eigenvector $\vec{1}$. Therefore, $RLR^T$ also has a $0$ eigenvalue, so we can therefore no longer find the desired $\Sigma_z$.  Since the last row of $R=\frac{1}{\sqrt{N}}\vec{1}$, all entries of the the last column of $LR^T$ are $0$. Therefore, all entries of the last column of $RLR^T$ are $0$. In other words, the dynamics of the first $N-1$ elements of $z$, i.e. $y$, do not depend on the last. This lets us write 
\begin{align*}
\dot{y}&=QLQ^Ty+Q\xi=\tilde{L}+Q\xi
\end{align*}
where $\tilde{L}=QLQ^T$. Another way to see this is to write 
\begin{align*}
 \dot{y}&=Q\dot{x}
 \\ &=QLx+Q\xi
 \\&=QL(x-\langle x\rangle \vec{1})+Q\xi \text{ since $L\vec{1}=\vec{0}$}
 \\&=QLQ^TQx+Q\xi
 \\&=\tilde{L}y+Q\xi.
\end{align*}
Again, by Claim \ref{lyapeq_derivation} and Theorem \ref{lyapeq_theorem}, we can now find $\lim_{t\to\infty}E[y(t)y(t)^T]=\Sigma_y$ by finding a solution to $\tilde{L}\Sigma_y+\Sigma_y\tilde{L}=-QQ^T$. 

To summarize, in the absence of an environmental signal, $\Sigma_y$ satisfies $\tilde{L}\Sigma_y+\Sigma_y\tilde{L}=-QQ^T$ where $\tilde{L}=QLQ^T$ and the presence of an environmental signal, $\Sigma_y$ can be found in the first $N-1$ rows and columns of $\Sigma_z$ where $\Sigma_z$ satisfies $\tilde{L}_\text{s}\Sigma_z+\Sigma_z\tilde{L}_\text{s}=-RR^T$ where $\tilde{L}_\text{s}=RL_\text{s}R^T$.


\subsection{Probability distribution at equilibrium }
A quite different way of modeling social interactions the method used by Bialek et al. in \cite{Bialek:2012fk}, which is to assume the opinions come to equilibrium and write down an expression for the probability of finding the opinion vector in every state. In statistical physics, a Hamiltonian describes the energy of a given state, $H(v)$, and the Boltzmann distribution describes a probability distribution in which the most energetic states are least likely. Analogously, we can write a function $H(v)$ that gives a high energy when the opinions are not in consensus and a low energy when the opinions are in consensus and a probability distribution of opinion vectors that favors less energetic states. In order to do this, we must use a symmetric interaction matrix, rather than the possibly asymmetric $\bar{A}$ we used above. Bialek et al. study a model in which opinions are a three-dimensional velocity vector and the birds can weight the opinions of their nearest neighbors with any value $J$. In order to simplify this model and make it directly comparable to the opinion dynamics model we described above, we only consider a scalar-valued opinion $x_i$ and we impose the condition that the weights given to the neighbors' opinions are inversely proportional to the number of neighbors. In particular, we define $S=\frac{1}{2}(A+A^T)$ and $P$ such that $P_{ij}=S_{ij}$ for $j\neq i$ and $P_{ii}=-\sum_kS_{ik}$. This allows us to write down a Hamiltonian
\begin{equation}
H(v)=-v^TPv
\end{equation}
where the negative sign makes opinions far from consensus give high energies and conversely. The probability distribution over opinion vectors $v$ then becomes 
\begin{equation}
P(v)=\frac{1}{Z}\exp\left(-\frac{1}{2}v^T(-P)v\right),
\end{equation}
where $Z=\int_\R^N\exp\left(-\frac{1}{2}H(v)\right)dv$. We can now use this probability distribution to find the  covariances and correlations between the birds' fluctuations around consensus.  If we let $V=\sum_iv_i/N=\langle v\rangle$ and $z=v-V\vec{1}$ so that $z$ represents the deviations in each bird's opinion from the consensus opinion. Then 
\begin{align*}
P(z,V)&=\frac{1}{Z}\exp\left(-\frac{1}{2}(z+V\vec{1})^T(-P)(z+V\vec{1})\right)
\\&=\frac{1}{Z}\exp\left(-\frac{1}{2}z^T(-P)z\right) \text{ since }P\vec{1}=P^T\vec{1}=\vec{0}.
\end{align*}
If $P$ were invertible, the covariance matrix would be given by $-P^{-1}$. However, since $P\vec{1}=\vec{0}$, one of the eigenvalues of $P$ is $0$ and it is therefore not invertible.
To proceed, we will write $z$ as a linear combination of eigenvectors of $P$. Since $P$ is symmetric and negative, there is an orthonormal set of eigenvectors $\{w_k\}$ such that $Pw_k=\lambda_kw_k$ with $w_N=\vec{1}$ and $\lambda_N=0$. If we let $W$ be a matrix whose columns are given by $\{w_k\}$ and $\Lambda$ be the diagonal matrix with $\{\lambda_k\}$ along the diagonal, then $W^T=W^{-1}$ and $P=W\Lambda W^T$ since $P$ is symmetric. If we write $z=\sum_kc_kw_k=W{c}$, then $c_N=0$ since $\sum_iz_i=0$. If we then denote the $N$ by $N-1$ matrix whose columns are given by $w_1,...w_{N-1}$ as $\tilde{W}$ and the diagonal matrix with $\lambda_1,\dots,\lambda_{N-1}$ along the diagonal as $\tilde{\Lambda}$, then $z=\tilde{W}\tilde{c}$ for a $\tilde{c}\in\R^{N-1}$ and
%and $\tilde{P}=\tilde{W}\tilde{\Lambda}\tilde{W}^T$, then 
\begin{align*}
P(z)&=\frac{1}{Z}\exp\left(-\frac{1}{2}z^T(-{P})z\right)
\\\Rightarrow P(\tilde{c})&=\frac{1}{Z}\exp\left(-\frac{1}{2}\tilde{c}^T\tilde{W}^T(-W\Lambda W^T)\tilde{W}\tilde{c}\right)
\\&=\frac{1}{Z}\exp\left(-\frac{1}{2}c^T\tilde{W}^T(-\tilde{W}\tilde{\Lambda}\tilde{W}^T)\tilde{W}c\right) 
\\&=\frac{1}{Z}\exp\left(-\frac{1}{2}\tilde{c}^T(-\tilde{\Lambda})\tilde{c}\right)  
\end{align*}
Therefore, $E[\langle c \rangle ]=\vec{0}$ and $E[\tilde{c}\tilde{c}^T]=(-\tilde{\Lambda})^{-1}$. Therefore, $C=E[zz^T]=E[\tilde{W}\tilde{c}\tilde{c}^T\tilde{W}^T]=-\tilde{W}\tilde{\Lambda}\tilde{W}^T$.  By dividing each entry $C_{ij}$ of the covariance matrix by $C_{ii}C_{jj}$, the covariance matrix can be used to find the correlation matrix.

The Hamiltonian framework, like the opinion dynamics framework, can also accommodate an environmental signal. In this case, the appropriate Hamiltonian is 
\begin{equation}
\begin{array}{lll}
H(v)&=-(v-\vec{1})^TP_\text{s}(v-\vec{1})
\end{array}
\end{equation}
where $P_\text{s}=P-B$.
It is clear that $H$ is minimized when $v=\vec{1}$. 
Again, we are interested in the deviations from the consensus opinion. As above, let $V=\sum_iv_i/N$ and $z=v-V\vec{1}$ so that $v=z+V\vec{1}$. 
\begin{align*}
P(v)&=\frac{1}{Z}\exp\left(-\frac{1}{2}(v-\vec{1})^T(-P_\text{s})(v-\vec{1}\right)
\\&=\frac{1}{Z}\exp\left(-\frac{1}{2}(z-(1-V)\vec{1})^T(-P_\text{s})(z-(1-V)\vec{1}\right)
\end{align*}
Since the subspace of $z\in\R^N$ such that $\sum_iz_i=0$ is $N-1$ dimensional, it will be easier to rotate $z$ into $\R^{N-1}$ and integrate over all of $\R^{N-1}$.  Again, we will use a rotation matrix $R\in\R^N\times\R^N$ such that the last row of $R$ is $\frac{1}{\sqrt{N}}\vec{1}$ and the other rows are normal and orthogonal to $\vec{1}$.
Above, we found a $Q\in\R^{N-1}\times\R^N$ such that each row of $Q$ was orthogonal to $\vec{1}$ and $R\in\R^N\times\R^N$ be the matrix with $\frac{1}{\sqrt{N}}\vec{1}^T$ as its last row and the first $N-1$ rows identical to the rows of $Q$. Then $R$ is a rotation matrix, so $RR^T=R^TR=I$ and $\det(R)=1$. This allows us to write $z-(1-V)\vec{1}=Q^TQv$ and $V=\frac{1}{N}\vec{1}^Tv$. If $y=Rz$, then $y_N=0$ since $\vec{1}^Tz=0$. Therefore, 
$$\Omega=\{z\in\R^N:\sum_iz_i=0\}=\{z\in\R^N:z\cdot\vec{1}=0\}=\{z\in\R^N:\exists \ y\in\R^N \text{ such that } z=R^Ty \text{ and } y_N=0\}.$$
Further, if $e_N=(0,0,\dots,0,1)^T$, $R^Te_N=q_N=\frac{1}{\sqrt{N}}\vec{1}$ so that $\vec{1}=\sqrt{N}R^Te_N$.This allows us to write
\begin{align*}
P(v)&=\frac{1}{Z}\exp\left(-\frac{1}{2}(R^Ty-(1-V)\sqrt{N}R^Te_N)^T(-P_\text{s})(R^Ty-(1-V)\sqrt{N}R^Te_N)\right)
\\&=\frac{1}{Z}\exp\left(-\frac{1}{2}(y-(1-V)\sqrt{N}e_N)^T(-RP_\text{s}R^T)(y-(1-V)\sqrt{N}e_N)\right)
\\&=\frac{1}{Z}\exp\left(-\frac{1}{2}(y+\sqrt{N}Ve_N-\sqrt{N}e_N)^T\Tilde{P}_\text{s}(y+\sqrt{N}Ve_N-\sqrt{N}e_N)\right)
\end{align*}
where $\Tilde{P}_\text{s}=-RP_\text{s}R^T$. To find $P(z)$ we can integrate out $V$. However, to consider this as a probability density function over $y$ and $V$, we have to consider the determinant of the transformation from $v$ to $y$ and $V$.  Combining the transformations from $v$ to $(z,V)$ and from $z$ to $(yV)$,
\begin{align*}
y=RQ^TQv=Qv \text{ and } V=\frac{1}{N}\vec{1}^Tv=\frac{1}{\sqrt{N}}R_{N\cdot }v
\end{align*}
so that the entire transformation can be represented by $\bar{R}=\left(\begin{array}{cc}Q \\1/\sqrt{N}R_{N\cdot }\end{array}\right)$, so that $\det(\bar{R})=\frac{1}{\sqrt{N}}\det(R)=\frac{1}{\sqrt{N}}$. Therefore, $P(y,V)=\sqrt{N}P(v)$ is the appropriate density function for $y$ and $V$. Finally, note that $Z=\sqrt{(2\pi)^N\det(-P_\text{s}^{-1})}$. This gives


%\begin{align*}
%P(z)&=\int_\R P(z,V)dV
%\\&=\int_\R \frac{1}{Z}\exp\left(-\frac{1}{2}((V-1)\sqrt{N}e_N+y)^T\Tilde{P}_\text{s}((V-1)\sqrt{N}e_N+y)\right)dV
%\\&=\frac{1}{Z}\exp(-\frac{1}{2}y^T\Tilde{P}_\text{s}y)\int_\R \exp\left(-\frac{1}{2}\left(\Tilde{P}_{\text{s}NN}N\left(V-1\right)^2+(V-1)\sqrt{N}\sum_{i=1}^{N-1}\Tilde{P}_{\text{s}Ni}y_i+(V-1)\sqrt{N}\sum_{i=1}^{N-1}y_i\Tilde{P}_{\text{s}iN}\right)\right)dV
%\\&=\frac{1}{Z}\exp(-\frac{1}{2}y^T\Tilde{P}_\text{s}y)\int_\R \exp\left(-\frac{1}{2}\left(\Tilde{P}_{\text{s}NN}N\left(V-1\right)^2+2(V-1)\sqrt{N}\sum_{i=1}^{N-1}\Tilde{P}_{\text{s}Ni}y_i\right)\right)dV \text{ since $\Tilde{P}_\text{s}$ is symmetric }
%\\&=\frac{1}{Z}\exp\left(-\frac{1}{2}\left(y^T\Tilde{P}_\text{s}y-\left(\sum_{i=1}^{N-1}\Tilde{P}_{\text{s}Ni}y_i/\sqrt{\Tilde{P}_{\text{s}NN}}\right)^2\right)\right)\times
%\\&\int_\R \exp\left(-\frac{1}{2}\left(\Tilde{P}_{\text{s}NN}N\left(V-1\right)^2+2(V-1)\sqrt{N}\sum_{i=1}^{N-1}\Tilde{P}_{\text{s}Ni}y_i+\left(\sum_{i=1}^{N-1}\Tilde{P}_{\text{s}Ni}y_i/\sqrt{\Tilde{P}_{\text{s}NN}}\right)^2\right)\right)dV 
%\\&=\frac{1}{Z}\exp\left(-\frac{1}{2}\left(y^T\Tilde{P}_\text{s}y-\left(\sum_i\Tilde{P}_{\text{s}Ni}y_i/\sqrt{\Tilde{P}_{\text{s}NN}}\right)^2\right)\right)\int_\R \exp\left(-\frac{1}{2}\left(\sqrt{\Tilde{P}_{\text{s}NN}}\sqrt{N}\left(V-1\right)+\sum_i\Tilde{P}_{\text{s}Ni}y_i/\sqrt{\Tilde{P}_{\text{s}NN}}\right)^2\right)dV
%\\&=\frac{1}{Z}\exp\left(-\frac{1}{2}\left(y^T\Tilde{P}_\text{s}y-\left(\sum_i\Tilde{P}_{\text{s}Ni}y_i/\sqrt{\Tilde{P}_{\text{s}NN}}\right)^2\right)\right)\int_\R \exp\left(-\frac{1}{2}\tilde{P}_{\text{s}NN}N\left(V-1+\frac{\sum_i\tilde{P}_{\text{s}Ni}y_i}{\tilde{P}_{\text{s}NN}\sqrt{N}}\right)^2\right)dV
%\\&=\frac{1}{Z}\exp\left(-\frac{1}{2}\left(y^T\tilde{P}_{\text{s}}y-\left(\sum_i\tilde{P}_{\text{s}Ni}y_i/\sqrt{\tilde{P}_{\text{s}NN}}\right)^2\right)\right)\times\sqrt{\frac{2\pi}{\tilde{P}_{\text{s}NN}N}}
%\\&=\frac{1}{Z}\exp\left(-\frac{1}{2}\left(y^T\tilde{P}_{\text{s}}y-\frac{1}{\tilde{P}_{\text{s}NN}}y^T\tilde{P}_{\text{s}\cdot N}\tilde{P}_{\text{s}N\cdot}y\right)\right)\times\sqrt{\frac{2\pi}{\tilde{P}_{\text{s}NN}N}} \text{ where $\tilde{P}_{\text{s}N\cdot}$ is the last row of $\tilde{P}_{\text{s}}$}
%\\&=\frac{1}{Z}\exp\left(-\frac{1}{2}\left(y^T(\tilde{P}_{\text{s}}-\frac{1}{\tilde{P}_{\text{s}NN}}\tilde{P}_{\text{s}\cdot N}\tilde{P}_{\text{s}N\cdot})y\right)\right)\times\sqrt{\frac{2\pi}{\tilde{P}_{\text{s}NN}N}}
%\\&=\frac{\sqrt{2\pi}}{\sqrt{(2\pi)^N\det(-P_\text{s}^{-1})\tilde{P}_{\text{s}NN}N}}\exp\left(-\frac{1}{2}\left(y^T(\tilde{P}_{\text{s}}-\frac{1}{\tilde{P}_{\text{s}NN}}\tilde{P}_{\text{s}\cdot N}\tilde{P}_{\text{s}N\cdot })y\right)\right)
%\\&=\frac{1}{\sqrt{(2\pi)^{N-1}\det(-P_\text{s}^{-1})\tilde{P}_{\text{s}NN}N}}\exp\left(-\frac{1}{2}z^TR^T(\tilde{P}_{\text{s}}-\frac{1}{\tilde{P}_{\text{s}NN}}\tilde{P}_{\text{s}\cdot N}\tilde{P}_{\text{s}N\cdot})Rz\right)
%\\&=\frac{1}{\sqrt{(2\pi)^{N-1}\det(-P_\text{s}^{-1})\tilde{P}_{\text{s}NN}N}}\exp\left(-\frac{1}{2}z^T(-P_\text{s}-\frac{1}{\tilde{P}_{\text{s}NN}}R^T\tilde{P}_{\text{s}\cdot N}\tilde{P}_{\text{s}N\cdot}R)z\right)
%%\\&=\frac{\sqrt{2\pi}\sqrt{R_{11}N}}{(2\pi)^{N/2}\det(-P_\text{s}^{-1})^{1/2}}\exp\left(-\frac{1}{2}z^T(-P_\text{s}-\frac{1}{R_{11}}Q'^{-1}R_{\cdot 1}R_{1\cdot}(Q'^{-1})^T)z\right)
%\\&=\frac{1}{\sqrt{(2\pi)^{N-1}\det(-P_\text{s}^{-1})\tilde{P}_{\text{s}NN}N}}\exp\left(-\frac{1}{2}z^T(-P_\text{s}-\frac{1}{\tilde{P}_{\text{s}NN}}P_\text{s}R^T_{\cdot N}R_{N\cdot}P_\text{s})z\right)
%\\&=\frac{1}{\sqrt{(2\pi)^{N-1}\det(-P_\text{s}^{-1})\tilde{P}_{\text{s}NN}N}}\exp\left(-\frac{1}{2}z^T(-P_\text{s}-\frac{1}{\tilde{P}_{\text{s}NN}}P_\text{s}(R_{N\cdot})^TR_{N\cdot}P_\text{s})z\right)
%%\\&=\frac{\sqrt{2\pi}\sqrt{R_{11}N}}{(2\pi)^{N/2}\det(-P_\text{s}^{-1})^{1/2}}\exp\left(-\frac{1}{2}z^T(-P_\text{s}-\frac{1}{R_{11}}P_\text{s}\Pi P_\text{s})z\right) \text{ where $\Pi_{ij}=1/N$ for all $i,j$}
%\\&=\frac{1}{\sqrt{(2\pi)^{N-1}\det(-P_\text{s}^{-1})\tilde{P}_{\text{s}NN}N}}\exp\left(-\frac{1}{2}z^T(-P_\text{s}-\frac{1}{\tilde{P}_{\text{s}NN}N}P_\text{s}\vec{1}\vec{1}^T P_\text{s})z\right) 
%\\&=\frac{1}{\sqrt{(2\pi)^{N-1}\det(-P_\text{s}^{-1})\tilde{P}_{\text{s}NN}N}}\exp\left(-\frac{1}{2}z^T(-P_\text{s}-\frac{1}{\tilde{P}_{\text{s}NN}N}B\vec{1}\vec{1}^T B)z\right)  \text{ since $P\vec{1}=\vec{0}$}
%\end{align*}

\begin{align*}
P(y,V)&=\sqrt{N}P(v)
\\&=\frac{\sqrt{N}}{Z}\exp\left(-\frac{1}{2}((V-1)\sqrt{N}e_N+y)^T\Tilde{P}_\text{s}((V-1)\sqrt{N}e_N+y)\right)
\\ \Rightarrow P(y)&=\int_\R P(y,V)dV
\\&=\int_\R \frac{\sqrt{N}}{Z}\exp\left(-\frac{1}{2}((V-1)\sqrt{N}e_N+y)^T\Tilde{P}_\text{s}((V-1)\sqrt{N}e_N+y)\right)dV
\\&=\frac{\sqrt{N}}{Z}\exp\left(-\frac{1}{2}y^T\Tilde{P}_\text{s}y\right)\int_\R \exp\left(-\frac{1}{2}\left(\Tilde{P}_{\text{s}NN}N\left(V-1\right)^2+(V-1)\sqrt{N}\tilde{P}_{\text{s}N\cdot}y+(V-1)\sqrt{N}y^T\tilde{P}_{\text{s}\cdot N}\right)\right)dV
\\&=\frac{\sqrt{N}}{Z}\exp\left(-\frac{1}{2}y^T\Tilde{P}_\text{s}y\right)\int_\R \exp\left(-\frac{1}{2}\left(\Tilde{P}_{\text{s}NN}N\left(V-1\right)^2+2(V-1)\sqrt{N}\tilde{P}_{\text{s}N\cdot}y\right)\right)dV \text{ since $\Tilde{P}_\text{s}$ is symmetric }
\\&=\frac{\sqrt{N}}{Z}\exp\left(-\frac{1}{2}\left(y^T\Tilde{P}_\text{s}y-\left(\frac{\tilde{P}_{\text{s}N\cdot}y}{\sqrt{\tilde{P}_\text{s}NN}}\right)^2\right)\right)\times
\\&\int_\R \exp\left(-\frac{1}{2}\left(\Tilde{P}_{\text{s}NN}N\left(V-1\right)^2+2(V-1)\sqrt{N}\tilde{P}_{\text{s}N\cdot}y+\left(\frac{\tilde{P}_{\text{s}N\cdot}y}{\sqrt{\Tilde{P}_{\text{s}NN}}}\right)^2\right)\right)dV 
\\&=\frac{\sqrt{N}}{Z}\exp\left(-\frac{1}{2}\left(y^T\left(\tilde{P}_{\text{s}}-\frac{1}{\tilde{P}_{\text{s}NN}}\tilde{P}_{\text{s}\cdot N}\tilde{P}_{\text{s}N\cdot}\right)y\right)\right)\int_\R \exp\left(-\frac{1}{2}\left(\sqrt{\Tilde{P}_{\text{s}NN}N}\left(V-1\right)+\frac{\tilde{P}_{\text{s}N\cdot}y}{\sqrt{\Tilde{P}_{\text{s}NN}}}\right)^2\right)dV
\\&=\frac{\sqrt{N}}{Z}\exp\left(-\frac{1}{2}\left(y^T(\tilde{P}_{\text{s}}-\frac{1}{\tilde{P}_{\text{s}NN}}\tilde{P}_{\text{s}\cdot N}\tilde{P}_{\text{s}N\cdot})y\right)\right)\int_\R \exp\left(-\frac{1}{2}\tilde{P}_{\text{s}NN}N\left(V-1+\frac{\sum_i\tilde{P}_{\text{s}Ni}y_i}{\tilde{P}_{\text{s}NN}\sqrt{N}}\right)^2\right)dV
\\&=\frac{\sqrt{N}}{Z}\exp\left(-\frac{1}{2}\left(y^T(\tilde{P}_{\text{s}}-\frac{1}{\tilde{P}_{\text{s}NN}}\tilde{P}_{\text{s}\cdot N}\tilde{P}_{\text{s}N\cdot})y\right)\right)\times\sqrt{\frac{2\pi}{\tilde{P}_{\text{s}NN}N}}
%\\&=\frac{\sqrt{2\pi}}{\sqrt{(2\pi)^N\det(-P_\text{s}^{-1})\tilde{P}_{\text{s}NN}}}\exp\left(-\frac{1}{2}\left(y^T(\tilde{P}_{\text{s}}-\frac{1}{\tilde{P}_{\text{s}NN}}\tilde{P}_{\text{s}\cdot N}\tilde{P}_{\text{s}N\cdot })y\right)\right)
\\&=\frac{1}{\sqrt{(2\pi)^{N-1}\det(-P_\text{s}^{-1})\tilde{P}_{\text{s}NN}}}\exp\left(-\frac{1}{2}\left(y^T(\tilde{P}_{\text{s}}-\frac{1}{\tilde{P}_{\text{s}NN}}\tilde{P}_{\text{s}\cdot N}\tilde{P}_{\text{s}N\cdot })y\right)\right)
\end{align*}



Therefore, $E[y_i]=0$ for all $i$ and $E[yy^T]=(\tilde{P}_{\text{s}}-\frac{1}{\tilde{P}_{\text{s}NN}}\tilde{P}_{\text{s}\cdot N}\tilde{P}_{\text{s}N\cdot })^{-1}$. Then, 
\begin{equation*}
E[zz^T]=R^T\left(\tilde{P}_{\text{s}}-\frac{1}{\tilde{P}_{\text{s}NN}}\tilde{P}_{\text{s}\cdot N}\tilde{P}_{\text{s}N\cdot }\right)^{-1}R=\left(P_\text{s}-\frac{1}{\tilde{P}_{\text{s}NN}N}B\vec{1}\vec{1}^TB\right)^{-1}
\end{equation*}
Let $N_\text{I}$ be the number of birds that can perceive the signal, i.e. $N_\text{I}$ is the number of non-zero elements along the diagonal of $B$. Then, $\tilde{P}_{\text{s}NN}=\frac{b}{N}N_\text{I}$. 
Finally, we find that 
\begin{equation}
E[zz^T]=\left(P_\text{s}-\frac{1}{bN_\text{I}}B\vec{1}\vec{1}^TB\right)^{-1}
\end{equation}
Now, let $P'=P_\text{s}+\frac{1}{bN_\text{I}}B\vec{1}\vec{1}^TB$. Then $P'\vec{1}=-B\vec{1}-\frac{1}{bN_\text{I}}B\vec{1}\vec{1}^TB\vec{1}=-B\vec{1}-\frac{bN_\text{I}}{bN_\text{I}}B\vec{1}=\vec{0}$, so we cannot invert $-P'$ to find the covariance matrix. However, as above, we can discard the $0$ eigenvector since we know that $z\cdot \vec{1}=0$. This gives us $E[(v_i-\frac{\sum_kv_k}{N})(v_j-\frac{\sum_kv_k}{N})]$.

\subsection{Connecting two frameworks }
To find $\mathscr{H}$, we found the matrix $\Sigma$ such that 
\begin{equation}
QLQ^T\Sigma+\Sigma QLQ^T=-I.
\end{equation}
If the vector $v$ is described by the dynamics $\dot{v}=Dv$, with $H(v)=-v^TPv$, then
\begin{align*}
\dot{H}&=-\dot{v}^TPv-v^TP\dot{v}
\\&=-v^TD^TPv-v^TPDv
\\&=-v^T(D^TP+PD)v.
\end{align*}
Therefore, $H(v)$ decreases at all $v$ if and only if $D^TP+PD$ is a positive matrix.
%MORE?!?

%Specifically, an individual's opinion $\sigma_i$ can be either $1$ or $-1$ and $\sigma_i$ switches with probability $w_i(\sigma_i)=\frac{1}{2}(-\sigma_i\sum_jA_{ij}\sigma_j )(1-\beta_i\sigma_i)$ where $A_{ij}$ indicates the strength of the connection from $j$ to $i$ and $\beta_i$ indicates the strength of the environmental stimulus perceived by $i$.  The stochastic dynamics of the opinions can be written down with a master equation \cite{Glauber:1963fk}.  The vector $\vec{v}(t)$ of expected values, $q_i(t)=\langle \sigma_i(t)\rangle$, satisfies the following differential equations: \cite{Glauber:1963fk}
%\begin{equation*}
%\frac{d\vec{v}}{dt}=M\vec{v}+\vec{\beta}^T(\vec{1}-\vec{v}).
%\end{equation*}
%We simplify this by writing $\frac{d\vec{v}}{dt}=A\vec{v}+\vec{\beta}^T\vec{1}$, where $A=M-B$ and $B$ is a diagonal matrix with the values of $\vec{\beta}$ along the diagonal. 
%
%
%
%
%
%
%
%
%
%
%
%

\begin{table}
\caption{ \label{matrices} Table of matrices and variables used in the text.}
\ra{1.3}
\begin{tabular}{@{}lllll@{}}
$A$ & adjacency matrix & $A_{ij}=\left\{\begin{array}{l l l}
1 & \text{ if $i$ pays attention to $j$ }
\\0 & \text{ otherwise }
\end{array}\right.$
\\$\bar{A}$ & normalized adjacency matrix & $\bar{A}_{ij}=\left\{\begin{array}{l l l }
\frac{1}{K_i} & \text{ if $i$ pays attention to $j$, where $K_i=$ \# of birds $i$ pays attention to}
\\ 0 & \text{ otherwise }
\end{array}\right. $
\\$B$ & diagonal signal matrix & $B_{ij}=\left\{\begin{array}{l l l}
b & \text{ if $i=j$ and $i$ perceives the environmental signal }
\\0 & \text{ otherwise }
\end{array}\right. $
\\$L$ & opinion dynamics w/o environmental signal &$L=\bar{A}-I$
\\$L_\text{s}$ & opinion dynamics with environmental signal & $L_\text{s}=L-B$
\\$\tilde{L}$ & dynamics of deviations from consensus w/o signal & $\tilde{L}=QLQ^T$
\\$\tilde{L}_\text{s}$ & dynamics of deviations from consensus w/ signal & $\tilde{L}=RL_\text{s}R^T$
\\$S$ & symmetrized interaction matrix & $S=\frac{1}{2}(\bar{A}+\bar{A}^T)$
\\$P$ & energy interactions w/o environmental signal & $P_{ij}=\left\{\begin{array}{lll}
S_{ij} & \text{ for } j\neq i
\\-\sum_kS_{ik} & \text{ for } j=i
\end{array}\right.
$
\\$P_\text{s}$ & energy interactions with environmental signal &  $P_\text{s}=P-B$
\\$\tilde{P}$ 
\\$\tilde{P}_\text{s}$
\\$Q$ & rotates opinion vector away from consensus & $Q\in \R^{N-1}\times\R^N \text{ such that } Q\vec{1}=\vec{0}, \ QQ^T=I_{N-1}, \text{ and } Q^TQ=I_N-\frac{1}{N}\vec{1}_N\vec{1}_N^T $
\\$R$ & full rotation matrix whose last row is $\frac{1}{\sqrt{N}}\vec{1}^T$ & $\left(\begin{array}{cc} Q \\ \frac{1}{\sqrt{N}}\vec{1}^T \end{array}\right)$
\\$r$ & $\mathscr{H}_2$ robustness & $\lim_{t\to\infty}\frac{1}{E[||y||]}$ where $y=Qx$
\\$W$ & 
\end{tabular}
\end{table}

\newpage
\section{Figures }
\begin{figure}
\includegraphics[width=6.83in]{/Users/eleanorbrush/Desktop/ESSfigure.pdf}
\caption{\label{ESS} The ESS number of neighbors is always higher if selection is due to predation than due to resources. When selection is due to predation, there can be multiple ESS strategies, which are a non-monotonic function of the radius of the signal. In (a), we show the (possibly multiple) ESS strategies as a function of the radius of the signal. In (b) we show the relative fitness of an invader as a function of the invader and resident strategies when $r=0.1$. The dots show the multiple ESSs. For (c) and (d), we choose a focal resident bird and consider the situation where neither the focal bird nor the invader has the environmental signal.The total of the stacked bars is the total probability that the invader will be predated when neither the focal bird nor the invader have the signal.  We further break down these cases into five cases: there is a single informed bird and neither focal bird, only the focal resident, only the invader, or both pay attention to it or there are multiple informed birds. The horizontal dashed line shows $1/N$, the probability of being predated if the strategies are neutral. In (c), the invader has one fewer neighbor than the resident. In (d), the invader has one more neighbor than the resident. Parameters:  in all panels $N=20$, $\beta=1$, $T=1$, in (b)-(d) $r=0.1$. 
}
\end{figure}

\begin{figure}
\includegraphics[width=6.83in]{/Users/eleanorbrush/Desktop/robustness_vs_ESS.pdf}
\caption{\label{corrESS}
%The correlation length of a homogeneous flock depends on both the strategy being used and the radius of the signal, but it depends much more strongly on the radius than the strategy. The ESS number of neighbors is an increasing function of correlation length of a homogeneous group. In (a), we show the average correlation length of a homogeneous flock as a function of the strategy being used and the radius of the signal. In (b), each point corresponds to a value of the radius of signal: the horizontal axis shows the correlation length of a flock paying attention to $N/2$ neighbors for that radius and the vertical axis shows the ESS number of neighbors for that radius. Parameters: $N=20$, $\beta=1$, $T=1$.
}
\end{figure}

\begin{figure}
%\includegraphics[width=6.83in]{/Users/eleanorbrush/Desktop/greedyoptneighbors.pdf}
\caption{\label{greedyopt} When the birds choose the optimal number of neighbors, given the strategies the rest of the flock are using, they settle on higher strategies when selection is due to predation than when selection is due to resources. Under selection due to predation, the $\mathscr{H}_2$ robustness and correlation length of the flock tend to decrease, whereas under selection due to resources, $\mathscr{H}_2$ robustness is relatively constant and correlation length tends to increase. The upper row shows results from implementing selection based to predation and the lower row shows results from implementing selection based on resources. In each panel, the  horizontal axis represents the number of times the birds are allowed to choose optimal strategies. The first column shows one example of how the birds' strategies change over time. The second column shows, for many initial conditions, how the $\mathscr{H}_2$ robustness changes over time. The third column shows, for many initial conditions, how the correlation length changes over time. Parameters:  $N=20$, $\beta=1$, $r=.1$, $T=1$. 
}
\end{figure}



\nocite{*}
\bibliographystyle{plain}
\bibliography{info_evo}

\end{document}


